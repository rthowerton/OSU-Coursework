\documentclass[10pt,draftclsnofoot,onecolumn]{IEEEtran}
\usepackage[utf8]{inputenc}
\usepackage[margin=.75in,letterpaper]{geometry}
\usepackage{bibentry}
\linespread{1.0}
\begin{document}
  \title{Group 15 Assignment 3 Report \vspace{\baselineskip}}
  \author{Ryan Howerton, Jinfeng Li, Vincenzo Piscitello \\
          CS444 Spring 2018 \vspace{10\baselineskip}}
  \maketitle
  \begin{abstract}
  	In the third assignment for Operating Systems II, we created a RAM Disk Driver which encrypts data as it is written and decrypts data when it is read. We accomplished this using the Linux Kernel Crypto API, and a sample simple block driver (SBD) found online, provided by the author of the book Linux Device Drivers (third edition). Our group studied both the Crypto API and the SBD to fully understand how to use them. For the cryptography portion, we decided to use the AES scheme with a randomly generated key (which we deemed usable considering that this driver should only ever be used on an expendable virtual machine). We wrote out our design, and created a detailed version control log and work log that keep track of our code edits and work time, respectively. Finally, we answer a few reflection questions on the work done.
  \end{abstract}
  \newpage
  %--------------------------------------------------------------------------
  % Design
  %--------------------------------------------------------------------------
	\section{Design}
    We used a simple block driver file found online as a base for our project, and dissected it to determine where to insert kernel cryptography commands to encrypt data as it is written, and decrypt data as it is read. To do this, we needed to create a crypto object, define a key, and initialize the object with a scheme. Finally, we decided that since this would be operating entirely on a virtual machine which is crushed and recreated each time, we would randomly generate a key using get\_random\_bytes() and storing it in a char array.
  %--------------------------------------------------------------------------
  % Version Control Log
  %--------------------------------------------------------------------------
    \section{Version Control Log}
\begin{center}
        \begin{tabular}{ | p{14cm} | }
        \hline
        
        {\bf Commits }\\ \hline
        {\bf commit} c2baed612cfa51aeca64677eda5bb3fd8e7b21a4\newline
        {\bf Author:} Vincenzo Piscitello \textless piscitev@os2.engr.oregonstate.edu \textgreater \newline
        {\bf Date:}   Sun May 24 17:30:00 2018 -0700\newline
        \newline  Changed driver filename and fixed buffer bug \\ \hline
		{\bf commit} fc1ffc0a6491c531d285196c3e91ba44ee04b09d \newline
        {\bf Author:} Ryan Howerton \textless howertor@oregonstate.edu \textgreater \newline
        {\bf Date:}   Sun May 17 18:00:00 2018 -0700\newline
        \newline Finished sbd.c \\ \hline
		{\bf commit} a29b80f7b56515b550e841414cdc3efb725fc258 \newline
        {\bf Author:} Ryan Howerton \textless howertor@oregonstate.edu \textgreater \newline
        {\bf Date:}   Sun May 17 16:30:00 2018 -0700\newline
        \newline Updated sbd.c \\ \hline
		{\bf commit} f8c90132c9c83186997f1663b843389521cb5e52\newline
        {\bf Author:} Vincenzo Piscitello \textless piscitev@os2.engr.oregonstate.edu \textgreater \newline
        {\bf Date:}   Sun May 16 16:26:00 2018 -0700\newline
        \newline  Added Makefile alterations \\ \hline
		{\bf commit} 448eb57c7712b69e92a70109e3b1c22bbcd20775\newline
        {\bf Author:} Vincenzo Piscitello \textless piscitev@os2.engr.oregonstate.edu \textgreater \newline
        {\bf Date:}   Sun May 16 15:45:00 2018 -0700\newline
        \newline  Added the crypto library include \\ \hline
        {\bf commit} 388d73054f44d3da3028baa01e16fabd087b7ca2 \newline
        {\bf Author:} Vincenzo Piscitello \textless piscitev@os2.engr.oregonstate.edu \textgreater \newline
        {\bf Date:}   Sun May 16 15:12:00 2018 -0700\newline
        \newline Added base SDB.c  \\ \hline
        {\bf commit} 1222d7e6c7804d9a7232654df3906500f653c2b8\newline
        {\bf Author:} Vincenzo Piscitello \textless piscitev@os2.engr.oregonstate.edu \textgreater \newline
        {\bf Date:}   Sat Apr 14 15:36:44 2018 -0700\newline
        \newline Initial Commit \\ \hline
        {\bf commit} 660613d1a4e94144490850b6c3d350331860fac4\newline
        {\bf Author:} Greg Kroah-Hartman \textless gregkh@linuxfoundation.org \textgreater \newline
        {\bf Date:}   Wed Mar 18 14:11:52 2015 +0100\newline
        \newline Linux 3.19.2  \\
        \hline
        \end{tabular}
    \end{center}
  %--------------------------------------------------------------------------
  % Work Log
  %--------------------------------------------------------------------------
    \section{Work Log}
      \begin{center}
        \begin{tabular}{ | p{4cm} | p{10cm} | }
          \hline
          Wednesday, May 16th & Downloaded the simple block device source code sbd.c. Added initialization code of crypto for memory encryption. Also, found out the algrothim used to encryption. Uploaded sbd.c to Github. \\ \hline
          Thursday, May 24th & Finished the encryption \& description parts and compiled sbd.c as a module in the VM. Found buffer bugs and fixed it. \\ \hline
          Friday, May 25th & Recompiled sbd.c and tested it in the VM make sure no error generated. Started writing parts. \\ \hline
          Sunday, May 27th & Finished writing work. Packaged everything up and submitted. \\ \hline
        \end{tabular}
      \end{center}
  %--------------------------------------------------------------------------
  % Question Answers
  %--------------------------------------------------------------------------
    \section{Questions}
      \begin{enumerate}
        \item What do you think the main point of this assignment is?
          \begin{itemize}
            \item The main point of the assignment is to practice using a poorly documented API and implement it in an established project.
          \end{itemize}
        \item How did you personally approach the problem? Design decisions, algorithm, etc.
          \begin{itemize}
            \item We began by looking into anything related to the provided search terms (SBD and 3.X). We found the simple block driver provided by the author of LDD3, and cross-referenced the Linux Kernel API to understand its operation. We then researched into the Kernel Crypto API to find applicable commands. Finally, we put everything together, added to the sbd, and tested it to make sure it worked.
          \end{itemize}
        \item How did you ensure your solution was correct? Testing details, for instance. Ensure this is written in a way that the TAs can follow to ensure correctness.
          \begin{itemize}
            \item We started by opening two terminal windows and sourcing the file environment-setup-i586-poky-linux in one of the windows. We ran make menuconfig to change the scheduler back to the standard one. We then ran make -j4 all to compile the kernel. After that completed, we ran the qemu virtual machine using the command qemu-system-i386 -redir tcp:5515::22 -nographic -kernel linux-yocto-3.19/arch/x86/boot/bzImage -drive file=core-image-lsb-sdk-qemux86.ext4 -enable-kvm -usb -localtime --no-reboot --append "root=/dev/hda rw console=ttyS0 debug", and logged in with the root account. Once the machine was running, in the other terminal window, we navigated to linux-yocto-3.19/drivers/block and ran make to build the sbd\_enc.ko file, and then used scp -P 5515 sbd\_enc.ko root@localhost:~ to copy the driver to the virtual machine. Finally, we just need to insert the driver by running insmod sbd\_enc.ko, and make sure that it properly installed by running lsmod.
          \end{itemize}
        \item What did you learn?
          \begin{itemize}
            \item We learned a good deal about the APIs both for the kernel and for kernel crypto. To use secure copy, we needed to learn a little about port redirection (as evidenced by the option -redir tcp:5515::22, which forces our port on the server to also listen on the ssh port, so as to let the secure copy communicate with the VM).
          \end{itemize}
      \end{enumerate}
\nocite{*}
\nobibliography{references.bib}
\bibliographystyle{alpha}
\end{document}