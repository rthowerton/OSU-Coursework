\documentclass[10pt,draftclsnofoot,onecolumn]{IEEEtran}
\usepackage[utf8]{inputenc}
\usepackage[margin=.75in,letterpaper]{geometry}
\usepackage{cite}
\usepackage{bibentry}
\linespread{1.0}

% Title Page
\title{CS444 Writing Assignment 1}
\author{Ryan Howerton \\ Spring 2018}

\begin{document}
	\maketitle
    \newpage
        %Introduce concept of processes and threads as explained by each source.
        A computer program is a set of instructions that, when executed, performs some predefined task. A program in execution is called a process. A process has at least one thread, and possibly more, executing the process's code. However, unless the computer in question has a multi-core processor, threads do not actually run at the same time, but are merely allowed enough execution time on the central processing unit (CPU) to give the illusion of concurrency. The entity behind this resource allocation is aptly called a CPU scheduler, and allows threads time on the CPU based on their importance to the continued functioning of the computer and its tasks. Every operating system (OS) provides this functionality to some extent, but the most widely known OS's are Linux, FreeBSD (a distribution of BSD, which MacOS is based off of), and Microsoft Windows.\par
        %How Linux approaches them.
        Linux is one of if not the most widely distributed operating system in the world. According to Klint Finley of Wired, over a third of all web servers run some form of Linux, 84\% of all smartphones use Android (which is based in Linux), and many more devices have it included \cite{finley_2017}. Linux does not differentiate between processes and threads. Linux therefore schedules threads, not processes. There are two types of threads: user threads and kernel threads. Only the kernel can start and stop kernel threads, and their primary function is to keep a system running. Threads in Linux are represented as objects, and are stored in a circular doubly-linked list for reference.\par
        As mentioned previously, while there may be more than one thread currently active, only one thread can be executing at any moment. This is where CPU scheduling comes into play. CPU scheduling is a function of the kernel. "Linux implements preemptive multitasking. Meaning, the scheduler decides when one process ceases running and the other begins...Preemptions are caused by timer interrupts" \cite{Ishkov_2015}. Tasks are split into two categories: I/O bound and processor bound. The latter heavily use whatever time is dedicated to them, the former block on I/O and so do not. "A scheduling policy in the system has to balance between the two types of processes, and to make sure that every task gets enough execution resources, with no visible effect on the performance of other jobs. To maximize CPU utilization and to guarantee fast response times, Linux tends to provide non-interactive processes with longer “uninterrupted” slices in a row, but to run them less frequently. I/O bound tasks, in turn, possess the processor more often, but for shorter periods of time" \cite{Ishkov_2015}.\par
        Processes are assigned priority values to determine how important they are to system function/process execution, and scheduling is controlled by policy groups that control how much time on-average processes get on the CPU \cite{Ishkov_2015}. %Real-Time Processes.
        %How FreeBSD approaches them.
        FreeBSD is a distribution of BSD, which in turn was a branch of UNIX developed for research purposes at the University of Berkeley. As mentioned previously, MacOS was developed from BSD, and so is a close relative of FreeBSD. While FreeBSD and Linux originated from the same source, they have slightly different approaches to process management.\par
        First, while Linux separates its user and kernel threads for different system uses, FreeBSD assigns its processes kernel-space resources. "Every thread running in the process has a corresponding kernel thread, with its own kernel stack that represents the user thread when it is executing in the kernel as a result of a system call, page fault, or signal delivery" \cite{mckusick_neville-neil_watson_mckusick_2015}. Because of this, a thread can operate in two modes: standard application execution, or calls to the operating system via system calls through the kernel.\par
        Second, while Linux operates almost exclusively on the thread level, FreeBSD maintains both a process structure and a thread structure. In this view, "the process structure contains information that must always remain resident in main memory, along with references to other structures that remain resident, whereas the thread structure tracks information that needs to be resident only when the process is executing such as its kernel run-time stack" \cite{mckusick_neville-neil_watson_mckusick_2015}.\par
        However, FreeBSD schedules processes and threads very similarly to Linux (ignoring the separation of the two objects). FreeBSD makes use of both scheduling priority values and policies. Like Linux, FreeBSD gives I/O bound tasks short periods of computation, followed by long periods of idling, and non-interactive processes longer periods of computation, but a lower priority so that other less active processes are allowed time. Like Linux, task priority is regularly recalculated to allow fair CPU time allocation, so "Jobs that use large amounts of CPU time sink rapidly to a low priority, whereas interactive jobs that are mostly inactive remain at a high priority so that, when they are ready to run, they will preempt the long-running lower-priority jobs" \cite{mckusick_neville-neil_watson_mckusick_2015}.\par
        %How Windows approaches them.
        %Conclusion(may not be necessary)
        
	\newpage
    \bibliographystyle{IEEEtran}
	\bibliography{references.bib}
\end{document}
