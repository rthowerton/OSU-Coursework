\documentclass[10pt,draftclsnofoot,onecolumn]{IEEEtran}
\usepackage[utf8]{inputenc}
\usepackage[margin=.75in,letterpaper]{geometry}
\usepackage{bibentry}
\linespread{1.0}

\title{Group 15 Assignment 2 Report \vspace{\baselineskip}}
\author{Ryan Howerton, Jinfeng Li, Vincenzo Piscitello \\
		CS444 Spring 2018 \vspace{10\baselineskip}}
        
\begin{document}
	\maketitle
    \begin{abstract}
  	In the second assignment for Operating Systems II, we edited the provided kernel code to implement a new 
    I/O scheduling algorithm. We decided to implement the C-LOOK algorithm, which continually services I/O 
    requests in order from the lowest targeted sector on the HDD to the highest. We started by looking at a 
    reference file for our SSTF scheduler, then creating a design based on the functions in that file. We logged 
    all of the changes we made to files and files added in a version control log. We created a work log 
    detailing when the group met and what we did when we met. Finally, we reflected on our process for 
    approaching the problem as well as the work we did.
  	\end{abstract}
    
    \newpage
	%-------------------------------------------------------------------------------
    % design
    %-------------------------------------------------------------------------------
    \section{Design}
    	Our design involved taking the noop-iosched.c file and changing it to sort new requests by the sector 
        they target, then service the requests in ascending order (C-LOOK algorithm). Specifically, we modified 
        the add\_request function to check the sector on the new request, search through the existing elevator 
        list to find the two elements with a lower and higher sector and insert the new request in between them.
        This way, as the elevator approaches each request, it will approach them from lowest to highest, and 
        once it hits the top of the list it will switch back down to the bottom.
    %-------------------------------------------------------------------------------
    % Version Control Log
    %-------------------------------------------------------------------------------
    \section{Version Control Log}
	\begin{center}
        \begin{tabular}{ | p{14cm} | }
        \hline
        {\bf Commits }\\ \hline
        {\bf commit} 2dfc59e0bd867fc86569bedbad944e0a90b7158d\newline
        {\bf Author:} Vincenzo Piscitello \textless piscitev@os2.engr.oregonstate.edu \textgreater \newline
        {\bf Date:}   Sun May 6 16:35:40 2018 -0700\newline
        \newline Ensure SSTF Scheduler is Commited  \\ \hline
        {\bf commit} 5ea65c92514eb9f2890b02f6ebd0eaaff17f7662\newline
        {\bf Author:} Vincenzo Piscitello \textless piscitev@os2.engr.oregonstate.edu \textgreater \newline
        {\bf Date:}   Sun May 6 16:02:05 2018 -0700\newline
        \newline Completed Assignment 2  \\ \hline
        {\bf commit} e736deabb4c7498afb804c84ec5dda0bb56ad72d\newline
        {\bf Author:} Vincenzo Piscitello \textless piscitev@os2.engr.oregonstate.edu \textgreater \newline
        {\bf Date:}   Sun Apr 15 15:42:00 2018 -0700\newline
        \newline Added Kernel Debugging Print Statements  \\ \hline
        {\bf commit} f0c5ea77098c2ded47d2b438509a628907489ca6\newline
        {\bf Author:} Vincenzo Piscitello \textless piscitev@os2.engr.oregonstate.edu \textgreater \newline
        {\bf Date:}   Sun Apr 15 15:30:00 2018 -0700\newline
        \newline Implemented C-LOOK Add Request  \\ \hline
        {\bf commit} 398f79b23165ecb79a0efeac34ca6b3787ec84ca\newline
        {\bf Author:} Vincenzo Piscitello \textless piscitev@os2.engr.oregonstate.edu \textgreater \newline
        {\bf Date:}   Sun May 5:19:20 2018 -0700\newline
        \newline Initialization Complete  \\ \hline
        {\bf commit} 1222d7e6c7804d9a7232654df3906500f653c2b8\newline
        {\bf Author:} Vincenzo Piscitello \textless piscitev@os2.engr.oregonstate.edu \textgreater \newline
        {\bf Date:}   Sat Apr 14 15:36:44 2018 -0700\newline
        \newline Initial Commit \\ \hline
        {\bf commit} 660613d1a4e94144490850b6c3d350331860fac4\newline
        {\bf Author:} Greg Kroah-Hartman \textless gregkh@linuxfoundation.org \textgreater \newline
        {\bf Date:}   Wed Mar 18 14:11:52 2015 +0100\newline
        \newline Linux 3.19.2  \\
        \hline
        \end{tabular}
    \end{center}

    %-------------------------------------------------------------------------------
    % Work Log
    %-------------------------------------------------------------------------------
    \section{Work Log}
    	\begin{center}
        \begin{tabular}{ | p{4cm} | p{10cm} | }
        \hline
        Saturday, April 29th & Researched and planned our method of implementing the C-LOOK algorithm.\\ \hline
        Saturday, May 5th & De-constructed each elevator operation to understand how it works then wrote pseudo-code to alter NO-OP to meet the C-LOOK requirements. After we completed our pseudo-code we implemented the C solution in sstf-iosched.c. We then recompiled menuconfig and re-compiled the linux kernel so that the SSTF scheduler could be used. \\ \hline
        Sunday, May 6th & Completed the homework document and tested the system to ensure it would build.\\ \hline
        \end{tabular}
    \end{center}
    %-------------------------------------------------------------------------------
    % Guided Question Answers
    %-------------------------------------------------------------------------------
    \section{Guided Question Answers}
    \begin{enumerate}
    	\item What do you think the main point of this assignment is?
    	\begin{itemize}
    		\item The main point of the assignment is to learn how to implement the LOOK algorithm in Linux I/O scheduler and configure it in VM, as a real life example for how kernel schedulers operate and are implemented.
        \end{itemize}
        \item How did you personally approach the problem?
        \begin{itemize}
        	\item We created a plan by referencing the noop scheduler, breaking down the elements of it on a whiteboard, and writing out our potential changes as pseudo-code. As soon as we thought we had a solution, we copied the noop file and renamed it for sstf, then implemented our changes with printk statements for debugging.
        \end{itemize}
        \item How did you ensure your solution was correct?
        \begin{itemize}
        	\item We compiled our kernel with the correct flags and modified Makefile, ensured that its default scheduler was the new sstf scheduler, then booted up the Qemu virtual machine and checked the kernel output for our print statements.
        \end{itemize}
        \item What did you learn?
        \begin{itemize}
        	\item First, we searched through the dependencies in the kernel to learn about potential tools at our disposal. We learned a great deal about all of the structures included in linux, and how to manipulate them. Then we delved deeper into the workings of the noop scheduler, to see what needed to be changed to implement the C-LOOK algorithm. We refreshed our memories on circular doubly-linked lists, and how to implement insertion sort on such a list.
           \par As for version control, we learned how to use the git diff command to create Linux patch file between two git commits and apply this patch file to source code without editing the source code directly. 
        \end{itemize}
        \item How should the TA evaluate your work?
        \begin{itemize}
        	\item The TA should evaluate our work by viewing the changes made through the patch file provided and ensuring the description of our solution is correct. To apply the changes use the bash git patch command passing in the "group15\_hw2.patch" file. 
        \end{itemize}
    \end{enumerate}

\nocite{*}
\nobibliography{bib.bib}
\bibliographystyle{alpha}
\end{document}
